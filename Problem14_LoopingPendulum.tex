%Mathematical Methods for Physics II Cheat Sheet
\documentclass[10pt,a4paper]{article}
\usepackage[UTF8]{ctex}
\usepackage{bm}
\usepackage{amsmath}
\usepackage{amssymb}
\begin{document}
\title{循环摆运动初期简单分析}
\author{陈稼霖}
\date{2019.6.27}
\maketitle
绞盘方程
\begin{equation}
\label{TensionRelation}
T_1=e^{\mu\theta}T_2
\end{equation}
对竖直悬挂重物$M$列出动力学方程
\begin{equation}
\label{MNewton2nd}
M\ddot{l}_1=Mg-T_1
\end{equation}
对拉起后荡下的重物$m$沿绳方向列出动力学方程
\begin{equation}
\label{mNewton2ndPara}
m(\ddot{l}_2-\dot{\theta}^2l_2)=-T_2-mg\cos\theta-m\ddot{\theta}l_2
\end{equation}
对拉起后荡下的重物$m$垂直绳方向列出动力学方程
\begin{equation}
\label{mNewton2ndPerp}
m(l_2\ddot{\theta}+2\dot{l}_2\dot{\theta})=mg\sin\theta-m\dot{\theta}^2R
\end{equation}
由绳长不变得关系
\begin{equation}
\label{RopeLengthConst}
l_1+l_2+R\theta=\text{const}\Longrightarrow\dot{l}_1+\dot{\theta}R+\dot{l}_2=0
\end{equation}
联立式(\ref{TensionRelation})与式(\ref{MNewton2nd})(从而消去$T_1$)可得
\begin{equation}
\label{noT1}
M\ddot{l}_1=-e^{\mu\theta}T_2+Mg
\end{equation}
再联立式(\ref{mNewton2ndPara})与式(\ref{noT1})(从而消去$T_2$)可得
\begin{equation}
\label{noT2}
e^{\mu\theta}m\dot{\theta}^2l_2-e^{\mu\theta}m\ddot{l}_2+M\ddot{l}_1=e^{\mu\theta}mg\cos\theta+e^{\mu\theta}m\ddot{\theta}l_2+Mg
\end{equation}
对式($\ref{RopeLengthConst}$)进一步求导得到
\begin{equation}
\label{dRopeLength}
\ddot{l}_1+R\ddot{\theta}+\ddot{l}_2=0
\end{equation}
我们发现
\[
\text{三个方程}\left\{\begin{array}{l}
\text{式(\ref{mNewton2ndPerp})}m(l_2\ddot{\theta}+2\dot{l}_2\dot{\theta})=mg\sin\theta-m\dot{\theta}^2R\\
\text{式(\ref{noT2})}e^{\mu\theta}m\dot{\theta}^2l_2-e^{\mu\theta}m\ddot{l}_2+M\ddot{l}_1=e^{\mu\theta}mg\cos\theta+e^{\mu\theta}m\ddot{\theta}l_2+Mg\\
\text{式(\ref{dRopeLength})}\ddot{l}_1+R\ddot{\theta}=-\ddot{l}_2
\end{array}\right.
\]
\[
\text{三个未知量}\left\{\begin{array}{l}
\theta\\
l_1\\
l_2
\end{array}\right.
\]
可解!\\
化成MATLAB$^\circledR$可求解的形式
\[
\left\{\begin{array}{l}
\ddot{\theta}=\cdots\\
\ddot{l}_1=\cdots\\
\ddot{l}_2=\cdots
\end{array}\right.
\]
代入MATLAB$^\circledR$中求解即可。
\end{document}